% Options for packages loaded elsewhere
\PassOptionsToPackage{unicode}{hyperref}
\PassOptionsToPackage{hyphens}{url}
%
\documentclass[
]{article}
\usepackage{amsmath,amssymb}
\usepackage{iftex}
\ifPDFTeX
  \usepackage[T1]{fontenc}
  \usepackage[utf8]{inputenc}
  \usepackage{textcomp} % provide euro and other symbols
\else % if luatex or xetex
  \usepackage{unicode-math} % this also loads fontspec
  \defaultfontfeatures{Scale=MatchLowercase}
  \defaultfontfeatures[\rmfamily]{Ligatures=TeX,Scale=1}
\fi
\usepackage{lmodern}
\ifPDFTeX\else
  % xetex/luatex font selection
\fi
% Use upquote if available, for straight quotes in verbatim environments
\IfFileExists{upquote.sty}{\usepackage{upquote}}{}
\IfFileExists{microtype.sty}{% use microtype if available
  \usepackage[]{microtype}
  \UseMicrotypeSet[protrusion]{basicmath} % disable protrusion for tt fonts
}{}
\makeatletter
\@ifundefined{KOMAClassName}{% if non-KOMA class
  \IfFileExists{parskip.sty}{%
    \usepackage{parskip}
  }{% else
    \setlength{\parindent}{0pt}
    \setlength{\parskip}{6pt plus 2pt minus 1pt}}
}{% if KOMA class
  \KOMAoptions{parskip=half}}
\makeatother
\usepackage{xcolor}
\usepackage[margin=1in]{geometry}
\usepackage{color}
\usepackage{fancyvrb}
\newcommand{\VerbBar}{|}
\newcommand{\VERB}{\Verb[commandchars=\\\{\}]}
\DefineVerbatimEnvironment{Highlighting}{Verbatim}{commandchars=\\\{\}}
% Add ',fontsize=\small' for more characters per line
\usepackage{framed}
\definecolor{shadecolor}{RGB}{248,248,248}
\newenvironment{Shaded}{\begin{snugshade}}{\end{snugshade}}
\newcommand{\AlertTok}[1]{\textcolor[rgb]{0.94,0.16,0.16}{#1}}
\newcommand{\AnnotationTok}[1]{\textcolor[rgb]{0.56,0.35,0.01}{\textbf{\textit{#1}}}}
\newcommand{\AttributeTok}[1]{\textcolor[rgb]{0.13,0.29,0.53}{#1}}
\newcommand{\BaseNTok}[1]{\textcolor[rgb]{0.00,0.00,0.81}{#1}}
\newcommand{\BuiltInTok}[1]{#1}
\newcommand{\CharTok}[1]{\textcolor[rgb]{0.31,0.60,0.02}{#1}}
\newcommand{\CommentTok}[1]{\textcolor[rgb]{0.56,0.35,0.01}{\textit{#1}}}
\newcommand{\CommentVarTok}[1]{\textcolor[rgb]{0.56,0.35,0.01}{\textbf{\textit{#1}}}}
\newcommand{\ConstantTok}[1]{\textcolor[rgb]{0.56,0.35,0.01}{#1}}
\newcommand{\ControlFlowTok}[1]{\textcolor[rgb]{0.13,0.29,0.53}{\textbf{#1}}}
\newcommand{\DataTypeTok}[1]{\textcolor[rgb]{0.13,0.29,0.53}{#1}}
\newcommand{\DecValTok}[1]{\textcolor[rgb]{0.00,0.00,0.81}{#1}}
\newcommand{\DocumentationTok}[1]{\textcolor[rgb]{0.56,0.35,0.01}{\textbf{\textit{#1}}}}
\newcommand{\ErrorTok}[1]{\textcolor[rgb]{0.64,0.00,0.00}{\textbf{#1}}}
\newcommand{\ExtensionTok}[1]{#1}
\newcommand{\FloatTok}[1]{\textcolor[rgb]{0.00,0.00,0.81}{#1}}
\newcommand{\FunctionTok}[1]{\textcolor[rgb]{0.13,0.29,0.53}{\textbf{#1}}}
\newcommand{\ImportTok}[1]{#1}
\newcommand{\InformationTok}[1]{\textcolor[rgb]{0.56,0.35,0.01}{\textbf{\textit{#1}}}}
\newcommand{\KeywordTok}[1]{\textcolor[rgb]{0.13,0.29,0.53}{\textbf{#1}}}
\newcommand{\NormalTok}[1]{#1}
\newcommand{\OperatorTok}[1]{\textcolor[rgb]{0.81,0.36,0.00}{\textbf{#1}}}
\newcommand{\OtherTok}[1]{\textcolor[rgb]{0.56,0.35,0.01}{#1}}
\newcommand{\PreprocessorTok}[1]{\textcolor[rgb]{0.56,0.35,0.01}{\textit{#1}}}
\newcommand{\RegionMarkerTok}[1]{#1}
\newcommand{\SpecialCharTok}[1]{\textcolor[rgb]{0.81,0.36,0.00}{\textbf{#1}}}
\newcommand{\SpecialStringTok}[1]{\textcolor[rgb]{0.31,0.60,0.02}{#1}}
\newcommand{\StringTok}[1]{\textcolor[rgb]{0.31,0.60,0.02}{#1}}
\newcommand{\VariableTok}[1]{\textcolor[rgb]{0.00,0.00,0.00}{#1}}
\newcommand{\VerbatimStringTok}[1]{\textcolor[rgb]{0.31,0.60,0.02}{#1}}
\newcommand{\WarningTok}[1]{\textcolor[rgb]{0.56,0.35,0.01}{\textbf{\textit{#1}}}}
\usepackage{graphicx}
\makeatletter
\def\maxwidth{\ifdim\Gin@nat@width>\linewidth\linewidth\else\Gin@nat@width\fi}
\def\maxheight{\ifdim\Gin@nat@height>\textheight\textheight\else\Gin@nat@height\fi}
\makeatother
% Scale images if necessary, so that they will not overflow the page
% margins by default, and it is still possible to overwrite the defaults
% using explicit options in \includegraphics[width, height, ...]{}
\setkeys{Gin}{width=\maxwidth,height=\maxheight,keepaspectratio}
% Set default figure placement to htbp
\makeatletter
\def\fps@figure{htbp}
\makeatother
\setlength{\emergencystretch}{3em} % prevent overfull lines
\providecommand{\tightlist}{%
  \setlength{\itemsep}{0pt}\setlength{\parskip}{0pt}}
\setcounter{secnumdepth}{-\maxdimen} % remove section numbering
\ifLuaTeX
  \usepackage{selnolig}  % disable illegal ligatures
\fi
\usepackage{bookmark}
\IfFileExists{xurl.sty}{\usepackage{xurl}}{} % add URL line breaks if available
\urlstyle{same}
\hypersetup{
  pdftitle={Basic R and how to read in data},
  hidelinks,
  pdfcreator={LaTeX via pandoc}}

\title{Basic R and how to read in data}
\author{}
\date{\vspace{-2.5em}}

\begin{document}
\maketitle

This guide is partly based on online material from Amy Willis, Kiirsti
Owen and Amelia McNamara, and the book ``R for Data Science'' by Hadley
Wickham and Garrett Grolemund. Thank you amazing R community!

\subsubsection{R as a calculator}\label{r-as-a-calculator}

In the Console window below, type: 2+2 and press Enter Also try:

\begin{Shaded}
\begin{Highlighting}[]
\DecValTok{2}\SpecialCharTok{\^{}}\DecValTok{5}
\end{Highlighting}
\end{Shaded}

\begin{verbatim}
## [1] 32
\end{verbatim}

\begin{Shaded}
\begin{Highlighting}[]
\DecValTok{3}\SpecialCharTok{/}\DecValTok{10}
\end{Highlighting}
\end{Shaded}

\begin{verbatim}
## [1] 0.3
\end{verbatim}

\begin{Shaded}
\begin{Highlighting}[]
\NormalTok{(}\DecValTok{3}\SpecialCharTok{+}\DecValTok{5}\NormalTok{)}\SpecialCharTok{\^{}}\DecValTok{2}
\end{Highlighting}
\end{Shaded}

\begin{verbatim}
## [1] 64
\end{verbatim}

\begin{Shaded}
\begin{Highlighting}[]
\FunctionTok{sqrt}\NormalTok{(}\DecValTok{4}\NormalTok{)}
\end{Highlighting}
\end{Shaded}

\begin{verbatim}
## [1] 2
\end{verbatim}

Tip: To run a line (or multiple lines) of code from a script without
typing them into the Console, select the line(s) you want to run and
press Ctrl+Enter (Command+Enter on a Mac)

\subsubsection{Objects}\label{objects}

R stores data as objects. You create new objects when you assign a value
to them using ``\textless-'':

\begin{Shaded}
\begin{Highlighting}[]
\NormalTok{x }\OtherTok{\textless{}{-}} \DecValTok{3}  \CommentTok{\# Check the "Environment" window!}
\end{Highlighting}
\end{Shaded}

Tip: use the R studio shortcut Alt+ - (Alt and the minus sign) to easily
create the assignment symbol \textless-

\begin{Shaded}
\begin{Highlighting}[]
\NormalTok{y }\OtherTok{\textless{}{-}} \DecValTok{6}
\NormalTok{x}\SpecialCharTok{+}\NormalTok{y}
\end{Highlighting}
\end{Shaded}

\begin{verbatim}
## [1] 9
\end{verbatim}

Tip: R is case sensitive so if you've defined your object as x, it will
not recognise (capital) X. Similarly, the function for square root is
sqrt, R will give you an error if you try to use Sqrt.

\subsubsection{Packages}\label{packages}

Packages extend the functionality of base R. They are distributed via
CRAN: the Comprehensive R Archive Network

To install a package, use: install.packages(``packagename'') You then
need to load it, using library(packagename)

We will be using a collection of packages called the Tidyverse:

\begin{Shaded}
\begin{Highlighting}[]
\FunctionTok{library}\NormalTok{(tidyverse)}
\end{Highlighting}
\end{Shaded}

\begin{verbatim}
## -- Attaching core tidyverse packages ------------------------ tidyverse 2.0.0 --
## v dplyr     1.1.4     v readr     2.1.5
## v forcats   1.0.0     v stringr   1.5.1
## v ggplot2   3.5.1     v tibble    3.2.1
## v lubridate 1.9.3     v tidyr     1.3.1
## v purrr     1.0.2     
## -- Conflicts ------------------------------------------ tidyverse_conflicts() --
## x dplyr::filter() masks stats::filter()
## x dplyr::lag()    masks stats::lag()
## i Use the conflicted package (<http://conflicted.r-lib.org/>) to force all conflicts to become errors
\end{verbatim}

When you load the tidyverse, you'll see a message about conflicts. As
there is an (increasingly) large number of packages in R, it is possible
to have functions with the same name in more than one package. The
message tells you that packages dplyr and stats both have a function
called filter and the one that will be used is the one from dplyr. It is
the one that was loaded last.

If you want to use a function from a particular package, you need to
include packagename:: before the name of the function.

In this example, you can use stats::filter() instead of just filter() to
use filter from the stats package.

\begin{Shaded}
\begin{Highlighting}[]
\FunctionTok{find}\NormalTok{(}\StringTok{"filter"}\NormalTok{) }\CommentTok{\# this shows you the packages a function belongs to, in order of priority}
\end{Highlighting}
\end{Shaded}

\begin{verbatim}
## [1] "package:dplyr" "package:stats"
\end{verbatim}

The tidyverse packages we will be using mostly in this course are readr
(for reading in data), dplyr (for transforming data) and ggplot2 (for
plotting).

\subsubsection{Functions}\label{functions}

When using the Tidyverse, you can call functions in two ways:

\begin{Shaded}
\begin{Highlighting}[]
\FunctionTok{sqrt}\NormalTok{(}\DecValTok{4}\NormalTok{) }\CommentTok{\# base R}
\end{Highlighting}
\end{Shaded}

\begin{verbatim}
## [1] 2
\end{verbatim}

\begin{Shaded}
\begin{Highlighting}[]
\DecValTok{4} \SpecialCharTok{\%\textgreater{}\%}
\NormalTok{  sqrt  }\CommentTok{\# "pipe" operator (you can read is as "and then...")}
\end{Highlighting}
\end{Shaded}

\begin{verbatim}
## [1] 2
\end{verbatim}

Tip: use the R Studio shortcut Ctrl + Sft + M to create the pipe
operator \%\textgreater\%

Tip: If you are not sure what a function does, type ?functionname in the
Console, e.g.~?sqrt

\subsubsection{Reading in data}\label{reading-in-data}

Before we read in our data, let's consider where we have saved our data
file. Since we want our code to be reusable (by us and other people),
the last thing we want is to include the location of the file in our
code, something like:

``C:/dimitra/data/datafile.csv''

The above would only work for me, and only for the particular computer
where folder ``dimitra'' contains a folder called ``data''.

To avoid these issues, we need to do two things:

\begin{enumerate}
\def\labelenumi{\arabic{enumi}.}
\tightlist
\item
  Use R projects. (I hope you are doing that already!) Save the data and
  R markdown file inside the R project. Exactly where you save your code
  doesn't matter, you just need to note the location of your data with
  respect to the .Rproj file.
\item
  Use the R package ``here''. ``Here'' points to the location of the
  .Rproj file (which is the working directory for your project), so you
  just need to add ``here'' in front of the relative path to your data
  file.
\end{enumerate}

For example, if your data file (a comma-separated value (csv) file) was
saved inside a ``data'' directory, you would say:

library(here)

fev\_data \textless- read\_csv(here(``data/fev.csv''))

To read in a file that is saved in the same directory as the .Rproj
file:

\begin{Shaded}
\begin{Highlighting}[]
\FunctionTok{library}\NormalTok{(here)}
\end{Highlighting}
\end{Shaded}

\begin{verbatim}
## here() starts at C:/Users/Yuluo Chen/OneDrive - University of Strathclyde/Documents/Jing/JW_PU5058_ABDN2024/read_in_data_week5
\end{verbatim}

\begin{Shaded}
\begin{Highlighting}[]
\NormalTok{fev\_data }\OtherTok{\textless{}{-}} \FunctionTok{read\_csv}\NormalTok{(}\FunctionTok{here}\NormalTok{(}\StringTok{"fev.csv"}\NormalTok{))}
\end{Highlighting}
\end{Shaded}

\begin{verbatim}
## Rows: 654 Columns: 7
\end{verbatim}

\begin{verbatim}
## -- Column specification --------------------------------------------------------
## Delimiter: ","
## dbl (7): seqnbr, subjid, age, fev, height, sex, smoke
## 
## i Use `spec()` to retrieve the full column specification for this data.
## i Specify the column types or set `show_col_types = FALSE` to quiet this message.
\end{verbatim}

(Remember to install the ``here'' package the first time.)

--\textgreater{} How would you use read\_csv with the pipe operator?

\begin{Shaded}
\begin{Highlighting}[]
\NormalTok{fev\_data }\OtherTok{\textless{}{-}} \StringTok{"fev.csv"} \SpecialCharTok{\%\textgreater{}\%}\NormalTok{ read\_csv}
\end{Highlighting}
\end{Shaded}

\begin{verbatim}
## Rows: 654 Columns: 7
## -- Column specification --------------------------------------------------------
## Delimiter: ","
## dbl (7): seqnbr, subjid, age, fev, height, sex, smoke
## 
## i Use `spec()` to retrieve the full column specification for this data.
## i Specify the column types or set `show_col_types = FALSE` to quiet this message.
\end{verbatim}

\begin{Shaded}
\begin{Highlighting}[]
\CommentTok{\# Add your code here!}
\end{Highlighting}
\end{Shaded}

Look at the top few rows of the data:

\begin{Shaded}
\begin{Highlighting}[]
\FunctionTok{head}\NormalTok{(fev\_data)}
\end{Highlighting}
\end{Shaded}

\begin{verbatim}
## # A tibble: 6 x 7
##   seqnbr subjid   age   fev height   sex smoke
##    <dbl>  <dbl> <dbl> <dbl>  <dbl> <dbl> <dbl>
## 1      1    301     9  1.71   57       0     0
## 2      2    451     8  1.72   67.5     0     0
## 3      3    501     7  1.72   54.5     0     0
## 4      4    642     9  1.56   53       1     0
## 5      5    901     9  1.90   57       1     0
## 6      6   1701     8  2.34   61       0     0
\end{verbatim}

fev\_data is a tibble - this is a tidyverse structure similar to a data
frame (from base R) but with some differences:

\begin{itemize}
\tightlist
\item
  default printing is shorter
\item
  tells you the column types (character, double, etc.)
\item
  doesn't change the types of inputs
\end{itemize}

Tip: if your data is in a Microsoft Excel spreadsheet, you will need a
different package to read it in, such as readxl. So you'll need:

install.packages(``readxl'')

library(readxl)

excel\_data \textless- read\_xlsx(filename, sheet = 1) \#(to read the
first sheet)

--\textgreater{} How would you read in a text file? (Check the data
import cheat sheet!) There is a text file in your dataset so you can
practice: psa.txt

\begin{Shaded}
\begin{Highlighting}[]
\NormalTok{psa\_data }\OtherTok{\textless{}{-}} \FunctionTok{read\_csv}\NormalTok{(}\StringTok{"psa.txt"}\NormalTok{) }\CommentTok{\# Add your code here!}
\end{Highlighting}
\end{Shaded}

\begin{verbatim}
## Rows: 50 Columns: 1
## -- Column specification --------------------------------------------------------
## Delimiter: ","
## chr (1): ptid   nadirpsa pretxpsa  ps        bss   grade      age    obstime...
## 
## i Use `spec()` to retrieve the full column specification for this data.
## i Specify the column types or set `show_col_types = FALSE` to quiet this message.
\end{verbatim}

--\textgreater{} Have a look at the ``Useful arguments'' section of the
data import cheat sheet. Use a few of them when you read in fev.csv and
look at the data, is that what you expected?

\begin{Shaded}
\begin{Highlighting}[]
\NormalTok{fev\_mm }\OtherTok{\textless{}{-}} \FunctionTok{read\_csv}\NormalTok{(}\StringTok{"fev.csv"}\NormalTok{,}\AttributeTok{skip=}\DecValTok{5}\NormalTok{) }\CommentTok{\# Add your code here!}
\end{Highlighting}
\end{Shaded}

\begin{verbatim}
## Rows: 649 Columns: 7
## -- Column specification --------------------------------------------------------
## Delimiter: ","
## dbl (7): 5, 901, 9, 1.895, 57, 1, 0
## 
## i Use `spec()` to retrieve the full column specification for this data.
## i Specify the column types or set `show_col_types = FALSE` to quiet this message.
\end{verbatim}

--\textgreater{} Apply the summary function to a tibble. What does it
do?

\begin{Shaded}
\begin{Highlighting}[]
\FunctionTok{summary}\NormalTok{(fev\_mm) }\CommentTok{\# Add your code here!}
\end{Highlighting}
\end{Shaded}

\begin{verbatim}
##        5            901              9              1.895             57       
##  Min.   :  6   Min.   :  201   Min.   : 3.000   Min.   :0.791   Min.   :46.00  
##  1st Qu.:168   1st Qu.:16151   1st Qu.: 8.000   1st Qu.:2.002   1st Qu.:57.50  
##  Median :330   Median :37241   Median :10.000   Median :2.560   Median :61.50  
##  Mean   :330   Mean   :37452   Mean   : 9.943   Mean   :2.644   Mean   :61.17  
##  3rd Qu.:492   3rd Qu.:54201   3rd Qu.:12.000   3rd Qu.:3.122   3rd Qu.:65.50  
##  Max.   :654   Max.   :90001   Max.   :19.000   Max.   :5.793   Max.   :74.00  
##        1                0         
##  Min.   :0.0000   Min.   :0.0000  
##  1st Qu.:0.0000   1st Qu.:0.0000  
##  Median :1.0000   Median :0.0000  
##  Mean   :0.5146   Mean   :0.1002  
##  3rd Qu.:1.0000   3rd Qu.:0.0000  
##  Max.   :1.0000   Max.   :1.0000
\end{verbatim}

\subsubsection{Operating on data:
columns}\label{operating-on-data-columns}

Individual columns are identified using the \$ symbol:

\begin{Shaded}
\begin{Highlighting}[]
\FunctionTok{head}\NormalTok{(fev\_data}\SpecialCharTok{$}\NormalTok{fev)}
\end{Highlighting}
\end{Shaded}

\begin{verbatim}
## [1] 1.708 1.724 1.720 1.558 1.895 2.336
\end{verbatim}

\begin{Shaded}
\begin{Highlighting}[]
\FunctionTok{summary}\NormalTok{(fev\_data}\SpecialCharTok{$}\NormalTok{fev)}
\end{Highlighting}
\end{Shaded}

\begin{verbatim}
##    Min. 1st Qu.  Median    Mean 3rd Qu.    Max. 
##   0.791   1.981   2.547   2.637   3.119   5.793
\end{verbatim}

\begin{Shaded}
\begin{Highlighting}[]
\FunctionTok{length}\NormalTok{(fev\_data}\SpecialCharTok{$}\NormalTok{fev)}
\end{Highlighting}
\end{Shaded}

\begin{verbatim}
## [1] 654
\end{verbatim}

Other useful functions for tibbles and data frames:

\begin{Shaded}
\begin{Highlighting}[]
\FunctionTok{names}\NormalTok{(fev\_data)}
\end{Highlighting}
\end{Shaded}

\begin{verbatim}
## [1] "seqnbr" "subjid" "age"    "fev"    "height" "sex"    "smoke"
\end{verbatim}

\begin{Shaded}
\begin{Highlighting}[]
\FunctionTok{dim}\NormalTok{(fev\_data)}
\end{Highlighting}
\end{Shaded}

\begin{verbatim}
## [1] 654   7
\end{verbatim}

Other useful functions for columns:

\begin{Shaded}
\begin{Highlighting}[]
\FunctionTok{max}\NormalTok{(fev\_data}\SpecialCharTok{$}\NormalTok{fev)}
\end{Highlighting}
\end{Shaded}

\begin{verbatim}
## [1] 5.793
\end{verbatim}

\begin{Shaded}
\begin{Highlighting}[]
\FunctionTok{mean}\NormalTok{(fev\_data}\SpecialCharTok{$}\NormalTok{fev)}
\end{Highlighting}
\end{Shaded}

\begin{verbatim}
## [1] 2.63678
\end{verbatim}

\begin{Shaded}
\begin{Highlighting}[]
\FunctionTok{sd}\NormalTok{(fev\_data}\SpecialCharTok{$}\NormalTok{fev)}
\end{Highlighting}
\end{Shaded}

\begin{verbatim}
## [1] 0.8670591
\end{verbatim}

\end{document}
